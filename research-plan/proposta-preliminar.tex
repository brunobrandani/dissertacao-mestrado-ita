\noindent
\textbf{Título:} Análise e Comparação de Ferramentas de Síntese de Alto Nível e
Linguagens de Programação no Contexto de Desenvolvimento de Sistemas Embarcados
Heterogêneos\\
\textbf{Nome:} Bruno Basso Brandani \\
\textbf{Curso:} Mestrado \\
\textbf{Programa:} Pós-Graduação em Engenharia Eletrônica e Computação (PG/EEC) \\
\textbf{Área de concentração:} Informática (EEC-I) \\
\textbf{Data:} \today \\
\textbf{Potencial orientador:} Prof. Dr. Denis S. Loubach \\

\centerline{\textbf{Proposta de Pesquisa}}
\centerline{}

% ----------------------------------------------
% Pergunta 1 - Área de Interesse
\noindent
\textbf{1. Área de Interesse}\\
\noindent
Área: Design de Sistemas Embarcados e Computação Reconfigurável\\
Linha de Pesquisa: Software e Sistemas de Informação\\
Palavras-chave: \emph{formal models of computation} (MoC); \emph{High-level synthesis} (HLS); \emph{Embedded systems}; \emph{Embedded domain-specific languages} (EDSL); \emph{Heterogeneous system}.\\

% ----------------------------------------------
% Pergunta 2 - Contextualização e Motivação
\noindent
\textbf{2. Contextualização e Motivação}\\
% \noindent
Sistemas embarcados estão atualmente presentes em diversas áreas, muitas vezes desempenhando papéis críticos, como por exemplo nos setores aeronáutico, automotivo, médico, entre outros. Além da confiabilidade exigida de tais sistemas, existe uma demanda alta por baixo consumo de energia, alta performance e custos reduzidos \cite{Horita}.

% \noindent
Do ponto de vista do hardware, está cada vez mais difícil o aprimoramento da
performance nas arquiteturas de computadores, principalmente com o fim da Escala
de Dennard, pelo \emph{dark silicon} e a considerável desaceleração da Lei de
Moore \cite{Loubach2022a}. A adoção de arquiteturas de domínio específico
(\emph{domain-specific architecture} - DSA), juntamente com linguagens de
domínio específico (\emph{domain-specific languages} - DSL), se torna uma
possível abordagem para contornar esse problema. Diferente de uma arquitetura de
propósito geral, DSA permite que o hardware se adapte e se otimize em tempo de
execução para um domínio específico \cite{Loubach2019a}.

% \noindent
Modelar e implementar tais sistemas não é trivial, principalmente quando se
trata de uma classe específica de hardware, os chamados sistemas
heterogêneos. Compostos por dispositivos programáveis (programáveis com
linguagens de alto nível) e dispositivos reconfiguráveis (reconfiguração do
hardware em nível de circuito eletrônico) \cite{Loubach2016a}, sistemas com
hardware heterogêneo proporcionam uma interessante combinação entre
flexibilidade e performance.

% \noindent
Considerando a complexidade destes sistemas, juntamente com a criticidade de suas aplicações e os níveis de desempenho e custo esperados, é interessante que seja adotada uma especificação de alto nível do sistema, utilizando modelos formais de computação e que possa ao mesmo tempo ser implementada em um hardware.

% \noindent
A abordagem por síntese de alto nível (\emph{high-level synthesis} - HLS)
permite que as etapas de design e especificação foquem nas funcionalidades e
requisitos que o sistema requer e ao mesmo tempo ofereça um modelo executável e
passível de síntese em linguagem de descrição de hardware (\emph{hardware
  description language} - HDL) de forma automatizada \cite{Loubach2022a}.\\

% ----------------------------------------------
% Pergunta 3 - Objetivo da Pesquisa
\noindent
\textbf{3. Objetivo da Pesquisa}\\
% \noindent
O objetivo da pesquisa é \textbf{estudar ferramentas e linguagens de alto nível
  que possibilitem síntese de alto nível}, i.e. a especificação do sistema
através de modelos formais de computação, a criação de um modelo executável e a
síntese em \textbf{hardware de forma automatizada}.

% \noindent
O estudo abordará os impactos que cada ferramenta e linguagem geram na performance do sistema, uma vez que tal abordagem aumenta o número de camadas de abstração entre o modelo de alto nível e o código gerado ao final do processo.\\

% ----------------------------------------------
% Pergunta 4 - Informações Complementares
\noindent
\textbf{4. Informações Complementares}\\
% \noindent
Um framework existente é o \emph{ForSyDe}. Ele inclui o \emph{ForSyDe-Deep}, uma
linguagem profundamente embarcada de domínio específico (\emph{deep} EDSL),
desenvolvida utilizando-se dos paradigmas da programação funcional usando
linguagem Haskell \cite{Loubach2022a}\cite{ForSyDe-Deep}.

% \noindent
Outro framework a ser analisado é o \emph{Ptolemy II}, o qual visa modelar e simular sistemas heterogêneos utilizando paradigmas da programação imperativa e orientada a objetos, tendo Java como sua base \cite{PtolemyII}.\\

% ----------------------------------------------
% Pergunta 5 - Aplicação dos Conhecimentos
\noindent
\textbf{5. Aplicação dos Conhecimentos}\\
% \noindent
É essencial garantir a corretude no desenvolvimento de sistemas críticos, tais quais sistemas médicos, automotivos e aeronáuticos. Estes sistemas tem seus desenvolvimentos pautados por normas, como no caso da RTCA DO-178C para o desenvolvimento de software aeronáutico. Para um software com condição de falha classificada como catastrófica (\emph{design assurance level} A), existem 71 objetivos associados ao processo, incluindo etapas de design, desenvolvimento, verificação e testes \cite{DO-178}.

% \noindent
Nestes sistemas, a síntese de alto nível pode contribuir com elevação da
qualidade do produto, ao mesmo tempo que permite o cumprimento dos objetivos
citados. Garantir que a fase de design e especificação de requisitos esteja
desassociada da implementação e síntese oferece maior maturidade e qualidade ao processo. A abordagem de \emph{correct-by-construction} é alcançada por meio do formalismo proporcionado pelos modelos de computação utilizados como base de HLS \cite{Edwards}.

% \noindent
Também é interessante o ganho de agilidade no processo de desenvolvimento que
tal abordagem propicia, uma vez que etapas realizadas manualmente (como por
exemplo implementação de requisitos em código manual) podem ser automatizadas
sem perda de qualidade. Por meio da qualificação de ferramentas que realizam HLS
é possível até mesmo eliminar em partes a intervenção humana em algumas etapas
do processo.


%%% Local Variables:
%%% mode: latex
%%% TeX-master: "main"
%%% End:
