\noindent
\textbf{Título:} \textcolor{red}{título preliminar para a pesquisa – até 2 linhas} \\
Análise e comparação de ferramentas e linguagens de programação no desenvolvimento de sistemas heterogêneos\\
\textbf{Nome:} Bruno Basso Brandani \\
\textbf{Curso:} Mestrado \\
\textbf{Programa:} Pós-Graduação em Engenharia Eletrônica e Computação (PG/EEC) \\
\textbf{Área de concentração:} Informática (EEC-I) \\
\textbf{Data:} \textcolor{red}{XX/Maio/2023} \\
\textbf{Potencial orientador:} Denis S. Loubach \\

\centerline{\textbf{Proposta de Pesquisa}}
\centerline{\textcolor{red}{[utilizar até 3 páginas] }}

\noindent
\textbf{1. Área de Interesse}\\
\noindent
\textcolor{red}{identificar a área e a linha de pesquisa em que se enquadra o tema proposto}
\\
Área: Design de Sistemas Embarcados e Computação Reconfigurável\\
Linha de Pesquisa: Desenvolvimento de Sistemas Heterogêneos\\

\noindent
\textbf{2. Contextualização e Motivação}\\
\noindent
\textcolor{red}{descrever o contexto da pesquisa e apresentar o problema que será abordado embasando-se em referências bibliográficas]}
\\

\noindent
\textbf{3. Objetivo da Pesquisa}\\
\noindent
\textcolor{red}{identificar o objetivo do trabalho de pesquisa proposto de forma coerente à solução do problema de pesquisa anteriormente especificado, podendo discriminá-lo em objetivo geral e objetivos específicos, para uma melhor compreensão, caso necessário}
\\

\noindent
\textbf{4. Informações Complementares}\\
\noindent
\textcolor{red}{Campo opcional onde podem ser apresentadas informações adicionais sobre a metodologia que se pretende desenvolver e os resultados esperados, identificado em referências bibliográficas consultadas ou testes/avaliações preliminares já obtidos/as pelo autor} 
\\

\noindent
\textbf{5. Aplicação dos Conhecimentos}\\
\noindent
\textcolor{red}{Campo opcional, destinado à descrição da visão de aplicação dos conhecimentos advindos do trabalho de pesquisa a ser realizado pelo candidato}
\\